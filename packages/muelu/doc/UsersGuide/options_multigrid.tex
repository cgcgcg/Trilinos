
\cbb{multigrid algorithm}{string}{"sa"}{Multigrid method. Possible values: see Table~\ref{t:mgs}.}
          
\cbb{semicoarsen: coarsen rate}{int}{3}{Rate at which to coarsen unknowns in the z direction.}
          
\cbb{semicoarsen: piecewise constant}{bool}{false}{whether or not to use piecewise constant interpolation when semicoarsening.}
          
\cbb{semicoarsen: piecewise linear}{bool}{false}{whether or not to use piecewise linear interpolation when semicoarsening.}
          
\cbb{semicoarsen: calculate nonsym restriction}{bool}{false}{whether or not to also calculate a restriction operator when semicoarsening (intended for non-symmetric systems).}
          
\cbb{sa: damping factor}{double}{1.33}{Damping factor for smoothed aggregation.}
          
\cbb{sa: use filtered matrix}{bool}{true}{Matrix to use for smoothing the tentative prolongator. The two options are: to use the original matrix, and to use the filtered matrix with filtering based on filtered graph used for aggregation.}
          
\cbb{interp: build coarse coordinates}{bool}{true}{If false, skip the calculation of coarse coordinates.}
          
\cbb{filtered matrix: use lumping}{bool}{true}{Lump (add to diagonal) dropped entries during the construction of a filtered matrix. This allows user to preserve constant nullspace.}
          
\cbb{filtered matrix: use root stencil}{bool}{false}{Use root-node based sparsification of the filtered
      matrix. This usually reduces operator complexity in the case of small aggregates.}
          
\cbb{filtered matrix: Dirichlet threshold}{double}{-1.0}{Dirichlet threshold to use for detecting zero
      diagonals in the filtered matrix (which get replaced with a one).  Any negative number disables
      the thresholding.}
          
\cbb{filtered matrix: reuse eigenvalue}{bool}{true}{Skip eigenvalue calculation during the construction of a filtered matrix by reusing eigenvalue estimate from the original matrix. This allows us to skip heavy computation, but may lead to poorer convergence.}
          
\cbb{emin: iterative method}{string}{"cg"}{Iterative method to use for energy minimization of initial prolongator in energy-minimization. Possible values: "cg" (conjugate gradient), "gmres" (generalized minimum residual), "sd" (steepest descent).}
          
\cbb{emin: num iterations}{int}{2}{Number of iterations to minimize initial prolongator energy in energy-minimization.}
          
\cbb{emin: num reuse iterations}{int}{1}{Number of iterations to minimize the reused prolongator energy in energy-minimization.}
          
\cbb{emin: pattern}{string}{"AkPtent"}{Sparsity pattern to use for energy minimization. Possible values: "AkPtent".}
          
\cbb{emin: pattern order}{int}{1}{Matrix order for the "AkPtent" pattern.}
          
\cbb{emin: use filtered matrix}{bool}{true}{Matrix to use for smoothing for energy minimization. The two options are: to use the original matrix, and to use the filtered matrix with filtering based on filtered graph used for aggregation.}
          
\cbb{emin: constraint type}{string}{nullspace}{Type of constraint to enforce. Valid values are "nullspace" and "maxwell".}
          
\cbb{emin: least squares solver type}{string}{direct}{Type of solver to use for least squares problem. Valid values are "direct" and "Belos". "Belos" only works for nullspace constraints.}
          